\chapter{Implementacija i korisničko sučelje}
		
		
		\section{Korištene tehnologije i alati}
		
			Za komunikaciju u timu korištene su aplikacije WhatsApp\footnote{\url{https://www.whatsapp.com/}} i Discord\footnote{\url{https://discord.com/}}. UML dijagrami izrađeni su pomoću alata Astah UML\footnote{\url{https://astah.net/products/astah-uml/}}, a zajedno s ostalim materijalima i informacijama u dokumentaciju su uneseni pomoću alata za uređivanje LaTeX dokumenata koji se zove Texmaker\footnote{\url{https://www.xm1math.net/texmaker/}} i beplatne verzije online alata iste vrste Overleaf\footnote{\url{https://www.overleaf.com/}}. Kao sustav za upravljanje izvornim kodom korišten je Git\footnote{\url{https://git-scm.com/}}, a udaljeni repozitorij nalazio se na web platformi GitHub\footnote{\url{https://github.com/}}. 

Kao razvojno okruženje korišteni su Visual Studio Code\footnote{\url{https://code.visualstudio.com/}}, uređivač koda koji je razvijen od strane kompanije Microsoft za operacijske sustave Windows, Linux i macOS, i IntelliJ IDEA\footnote{\url{https://www.jetbrains.com/idea/}}, integrirano razvojno okruženje (IDE) razvijeno od strane kompanije JetBrains. Backend je realiziran pomoću radnog okvira Node.js\footnote{\url{https://nodejs.org/}} koji omogućuje korištenje jezika JavaScript\footnote{\url{https://www.javascript.com/}} na serverskoj strani aplikacije. API je pisan u jeziku Java\footnote{\url{https://www.java.com/}} uz korištenje radnog okvira Java Spring Boot\footnote{\url{https://spring.io/projects/spring-boot/}} koji ima ugrađenu podršku za zadatke poput upravljanja konfiguracijom, pristupa podatcima u bazi podataka i intergracije sigurnosti u aplikaciju. Aplikacija pristupa MongoDB\footnote{\url{https://www.mongodb.com/}} bazi podataka za čiji je lakši pregled korišten alat MongoDB Compass\footnote{\url{https://www.mongodb.com/products/tools/compass}}. Frontend je raliziran pomoću biblioteke React\footnote{\url{https://react.dev/}}, također poznate kao React.js ili ReactJS, koja omogućuje pisanje HTML koda u JavaScript datotekama, što skupa s modularnom arhitekturom i komponentama koje pruža biblioteka, čini jednostavnim dinamičko generiranje korisničkih sučelja. Za lakše stvaranje nekih komponenti korisnočkog sučelja korišten je radni okvir Bootstrap\footnote{\url{https://getbootstrap.com/}} koji pruža HTML, CSS i JavaScript kod često potrebnih dijelova aplikacije.

Za testiranje API-ja korišten je alat Postman\footnote{\url{https://www.postman.com/}} koji omogućuje slanje različitih vrsta zahtjeva na server. Aplikacija je testirana pomoću alata Selenium WebDriver\footnote{\url{https://www.selenium.dev/documentation/webdriver/}} i JUnit\footnote{\url{https://junit.org/}}, koji omogućuju automatizaciju testiranja web aplikacija pomoću programskog sučelja. Interakcija Selenium WebDrivera s Google Chrome pretraživačem ostavrena alatom ChromeDriver\footnote{\url{https://chromedriver.chromium.org/}}.
			\eject 
		
	
		\section{Ispitivanje programskog rješenja}
			
			\subsection{Ispitivanje komponenti}
			\noindent \textbf{Testni slučaj 1: Provjera addEvent funkcije u Event servisu}
						\begin{itemize}
	
						\item[] \textbf{Ulaz: }
						\begin{packed_enum}
							\item Inicijalizacija događanja
							\item Dodavanje imena "Test Event" događanju 
							\item Pokretanje funkcije \textit{addEvent}
							\item Provjera podudaranja testnog imena s imenom novo nastalog događanja u bazi 
						
						\end{packed_enum}
						\item[] \textbf{Očekivani rezultat: }
						\begin{packed_enum}
							\item Imena će se podudarati
						
						\end{packed_enum}
						\item[] \textbf{Rezultat: }
						\begin{packed_enum}
							\item  Imena se podudaraju
						
						\end{packed_enum}
						\end{itemize}
						
						\begin{figure}[H]
							\includegraphics[width=\textwidth]{slike/IKTest1.PNG} %veličina u odnosu na širinu linije
							\caption{Prvi testni slučaj - addEvent}
							\label{fig:IKT1} %label mora biti drugaciji za svaku sliku
						\end{figure}
						\eject
						
						
						
			\noindent \textbf{Testni slučaj 2: Provjera editEvent funkcije u Event servisu}
						\begin{itemize}
	
						\item[] \textbf{Ulaz: }
						\begin{packed_enum}
							\item Inicijalizacija događanja
							\item Dodavanje imena "Original Event Name" i ID-a događanju 
							\item Spremanje događanja pomoću funkcije \textit{addEvent}
							\item Zadavanje novog imena "Updated Event Name" događanju
							\item Pokretanje funkcije \textit{editEvent}
							\item Usporedba uređenog imena sa imenom događanja u bazi
						
						\end{packed_enum}
						\item[] \textbf{Očekivani rezultat: }
						\begin{packed_enum}
							\item Imena će se podudarati
						
						\end{packed_enum}
						\item[] \textbf{Rezultat: }
						\begin{packed_enum}
							\item Događanje ima novo ime koje se podudara "Updated Event Name"
						
						\end{packed_enum}
						\end{itemize}
						
						\begin{figure}[H]
							\includegraphics[width=\textwidth]{slike/IKTest2.PNG} %veličina u odnosu na širinu linije
							\caption{Drugi testni slučaj - editEvent}
							\label{fig:IKT2} %label mora biti drugaciji za svaku sliku
						\end{figure}
						\eject
						
						
						
			\noindent \textbf{Testni slučaj 3: Provjera getUserById funkcije u User servisu}
						\begin{itemize}
	
						\item[] \textbf{Ulaz: }
						\begin{packed_enum}
							\item Zadavanje ID-a za pronalaženje
							\item Dohvaća se korisnik s tim ID-em izravno iz baze
							\item Pokretanje funkcije \textit{getUserById} za zadani ID
							\item Usporedba dohvaćenog korisnika izravno iz baze i korisnika dohvaćenog preko funkcije \textit{getUserById}
						
						\end{packed_enum}
						\item[] \textbf{Očekivani rezultat: }
						\begin{packed_enum}
							\item Korisnici su jednaki
						
						\end{packed_enum}
						\item[] \textbf{Rezultat: }
						\begin{packed_enum}
							\item Dohvaćeni korisnici se podudaraju
						
						\end{packed_enum}
						\end{itemize}
						
						\begin{figure}[H]
							\includegraphics[width=\textwidth]{slike/IKTest3.PNG} %veličina u odnosu na širinu linije
							\caption{Treći testni slučaj - getUserById}
							\label{fig:IKT3} %label mora biti drugaciji za svaku sliku
						\end{figure}
						\eject
						
						
						
			\noindent \textbf{Testni slučaj 4: Provjera getUserById funkcije u User servisu (neuspješno)}
						\begin{itemize}
	
						\item[] \textbf{Ulaz: }
						\begin{packed_enum}
							\item Zadavanje ID-a za pronalaženje
							\item Dohvaća se korisnik s tim ID-em izravno iz baze
							\item Pokretanje funkcije \textit{getUserById} za zadani ID
							\item Usporedba dohvaćenog korisnika izravno iz baze i korisnika dohvaćenog preko funkcije \textit{getUserById}
						
						\end{packed_enum}
						\item[] \textbf{Očekivani rezultat: }
						\begin{packed_enum}
							\item Uspoređuju se dvije prezne vrijednosti jer takav korisnik ne postoji
						
						\end{packed_enum}
						\item[] \textbf{Rezultat: }
						\begin{packed_enum}
							\item Test prolazi jer su obje vraćene vrijednosti prazne
							
						
						\end{packed_enum}
						\end{itemize}
						
						\begin{figure}[H]
							\includegraphics[width=\textwidth]{slike/IKTest4.PNG} %veličina u odnosu na širinu linije
							\caption{Četvrti testni slučaj - getUserById (nepostojeći)}
							\label{fig:IKT4} %label mora biti drugaciji za svaku sliku
						\end{figure}
						\eject
						
						
						
			\noindent \textbf{Testni slučaj 5: Provjera allEvents funkcije u Event servisu}
						\begin{itemize}
	
						\item[] \textbf{Ulaz: }
						\begin{packed_enum}
							\item Dohvaćanje svih događanja izravno iz baze
							\item Pozivanje funkcije \textit{allEvents}
							\item Usporedba dohvaćenih događanja iz baze s dohvaćenim događanjima preko \textit{allEvents} funkcije
						
						\end{packed_enum}
						\item[] \textbf{Očekivani rezultat: }
						\begin{packed_enum}
							\item Dobivena događanja se podudaraju
						
						\end{packed_enum}
						\item[] \textbf{Rezultat: }
						\begin{packed_enum}
							\item Popisi događanja se podudaraju
							
						
						\end{packed_enum}
						\end{itemize}
						
						\begin{figure}[H]
							\includegraphics[width=\textwidth]{slike/IKTest5.PNG} %veličina u odnosu na širinu linije
							\caption{Peti testni slučaj - allEvents}
							\label{fig:IKT5} %label mora biti drugaciji za svaku sliku
						\end{figure}
						\eject
						
						
						
			\noindent \textbf{Testni slučaj 6: Provjera getEventById funkcije u Event servisu}
						\begin{itemize}
	
						\item[] \textbf{Ulaz: }
						\begin{packed_enum}
							\item Dohvaćanje događanja preko ID-a izravno iz baze
							\item Pozivanje funkcije \textit{getEventById} za isti ID
							\item Usporedba dohvaćenog događanja iz baze s dohvaćenim događanjem preko \textit{getEventById} funkcije
						
						\end{packed_enum}
						\item[] \textbf{Očekivani rezultat: }
						\begin{packed_enum}
							\item Dobivena događanja se podudaraju
						
						\end{packed_enum}
						\item[] \textbf{Rezultat: }
						\begin{packed_enum}
							\item Događanja se podudaraju
							
						
						\end{packed_enum}
						\end{itemize}
						
						\begin{figure}[H]
							\includegraphics[width=\textwidth]{slike/IKTest6.PNG} %veličina u odnosu na širinu linije
							\caption{Šesti testni slučaj - getUserById}
							\label{fig:IKT6} %label mora biti drugaciji za svaku sliku
						\end{figure}
						\eject
						
				\noindent \textbf{Uspješnost testova - Ispitivanje komponenata}
						\begin{figure}[H]
							\includegraphics[width=\textwidth]{slike/IKTestovi.PNG} %veličina u odnosu na širinu linije
							\caption{Svi testovi komponenti}
							\label{fig:IKTS} %label mora biti drugaciji za svaku sliku
						\end{figure}
						\eject
						
						
			
			
			
			\subsection{Ispitivanje sustava}
			
				\noindent \textbf{Testni slučaj 1: Provjera \textit{Login} funkcionalnosti (UC3)}
						\begin{itemize}
	
						\item[] \textbf{Ulaz: }
						\begin{packed_enum}
							\item Otvaranje login stranice
							\item Upisivanje podataka za prijavu
							\item Unos podataka u aplikaciju
						
						\end{packed_enum}
						\item[] \textbf{Očekivani rezultat: }
						\begin{packed_enum}
							\item Dobiven novi url kojim se dolazi na prijavljeno sučelje
						
						\end{packed_enum}
						\item[] \textbf{Rezultat: }
						\begin{packed_enum}
							\item Test uspješan
							
						
						\end{packed_enum}
						\end{itemize}
						
						\begin{figure}[H]
							\includegraphics[width=\textwidth]{slike/IsTest1.PNG} %veličina u odnosu na širinu linije
							\caption{Prvi testni slučaj - Login}
							\label{fig:IST1} %label mora biti drugaciji za svaku sliku
						\end{figure}
						\eject
						
						
						
				\noindent \textbf{Testni slučaj 2: Provjera \textit{Login} funkcionalnosti (pogrešni podatci) (UC3)}
						\begin{itemize}
	
						\item[] \textbf{Ulaz: }
						\begin{packed_enum}
							\item Otvaranje login stranice
							\item Upisivanje podataka za prijavu
							\item Unos podataka u aplikaciju
						
						\end{packed_enum}
						\item[] \textbf{Očekivani rezultat: }
						\begin{packed_enum}
							\item Nije dobiven traženi url i javlja se poruka o pogrešno unesenim podatcima
						
						\end{packed_enum}
						\item[] \textbf{Rezultat: }
						\begin{packed_enum}
							\item Test uspješan
							
						
						\end{packed_enum}
						\end{itemize}
						
						\begin{figure}[H]
							\includegraphics[width=\textwidth]{slike/IsTest2.PNG} %veličina u odnosu na širinu linije
							\caption{Drugi testni slučaj - Login (pogrešni podatci)}
							\label{fig:IST2} %label mora biti drugaciji za svaku sliku
						\end{figure}
						\eject
						
						
						
				\noindent \textbf{Testni slučaj 3: Provjera funkcionalnosti kreiranja događanja (UC14)}
						\begin{itemize}
	
						\item[] \textbf{Ulaz: }
						\begin{packed_enum}
							\item Otvaranje login stranice
							\item Prijava
							\item Otvaranje padajućeg izbornika 
							\item Odabir opcije stvaranja događanja
							\item Upisivanje podataka o događanju
							\item Unos podataka u aplikaciju
							
						
						\end{packed_enum}
						\item[] \textbf{Očekivani rezultat: }
						\begin{packed_enum}
							\item Otvaranje \textit{pop up} prozora s potvrdom o stvaranju događanja
						
						\end{packed_enum}
						\item[] \textbf{Rezultat: }
						\begin{packed_enum}
							\item Test uspješan
							
						
						\end{packed_enum}
						\end{itemize}
						
						\begin{figure}[H]
							\includegraphics[width=\textwidth]{slike/IsTest3.PNG} %veličina u odnosu na širinu linije
							\caption{Treći testni slučaj - Kreiranje događanja}
							\label{fig:IST3} %label mora biti drugaciji za svaku sliku
						\end{figure}
						\eject
						
						
						
				\noindent \textbf{Testni slučaj 4: Provjera funkcionalnosti promjene lozinke (UC5)}
						\begin{itemize}
	
						\item[] \textbf{Ulaz: }
						\begin{packed_enum}
							\item Otvaranje login stranice
							\item Prijava
							\item Otvaranje padajućeg izbornika 
							\item Odabir opcije pregleda profila
							\item Klika na gumb za promjenu lozinke
							\item Upisivanje stare i nove lozinke
							\item Unos podataka u aplikaciju
							
						
						\end{packed_enum}
						\item[] \textbf{Očekivani rezultat: }
						\begin{packed_enum}
							\item Otvaranje \textit{pop up} prozora s potvrdom o promjeni lozinke
						
						\end{packed_enum}
						\item[] \textbf{Rezultat: }
						\begin{packed_enum}
							\item Test uspješan
							
						
						\end{packed_enum}
						\end{itemize}
						
						\begin{figure}[H]
							\includegraphics[width=\textwidth]{slike/IsTest4.PNG} %veličina u odnosu na širinu linije
							\caption{Četvrti testni slučaj - Promjena lozinke}
							\label{fig:IST4} %label mora biti drugaciji za svaku sliku
						\end{figure}
						\eject
						
						
						
				\noindent \textbf{Testni slučaj 5: Provjera funkcionalnosti promjene prikaza profila}
						\begin{itemize}
	
						\item[] \textbf{Ulaz: }
						\begin{packed_enum}
							\item Otvaranje login stranice
							\item Prijava
							\item Otvaranje padajućeg izbornika 
							\item Odabir opcije pregleda profila
							\item Klik na gumb za otvaranje prikaza javnog profila
							
						
						\end{packed_enum}
						\item[] \textbf{Očekivani rezultat: }
						\begin{packed_enum}
							\item Dobiven novi url kojim se dolazi na prikaz javnog profila
						
						\end{packed_enum}
						\item[] \textbf{Rezultat: }
						\begin{packed_enum}
							\item Test uspješan
							
						
						\end{packed_enum}
						\end{itemize}
						
						\begin{figure}[H]
							\includegraphics[width=\textwidth]{slike/IsTest5.PNG} %veličina u odnosu na širinu linije
							\caption{Peti testni slučaj - Otvaranje prikaza javnog profila}
							\label{fig:IST5} %label mora biti drugaciji za svaku sliku
						\end{figure}
						\eject
						
						
						\noindent \textbf{Uspješnost testova - Ispitivanje sustava}
						\begin{figure}[H]
							\includegraphics[width=\textwidth]{slike/ISTestovi.PNG} %veličina u odnosu na širinu linije
							\caption{Svi testovi sustava}
							\label{fig:ISTS} %label mora biti drugaciji za svaku sliku
						\end{figure}
						\eject
				
			
		
		
		\section{Dijagram razmještaja}
		Dijagram razmještaja (\ref{fig:DR}) prikazuje fizičku arhitekturu i konfiguraciju razmještaja programskog sustava. Poslužiteljsko računalo sadrži web poslužitelj na kojem je web aplikcaija i poslužitelj baze podataka koji omogućava pristup bazi. Klijent se na sustav spaja pomoću vlastitog osobnog računala koje uspostavlja komunikaciju preko HTTP protokola s poslužiteljskim računalom.  
			
			 \begin{figure}[H]
			\includegraphics[width=\textwidth]{slike/Dijagram razmjestaja.PNG} %veličina u odnosu na širinu linije
			\caption{Dijagram razmještaja}
			\label{fig:DR} %label mora biti drugaciji za svaku sliku
		\end{figure}
			\eject 
		
		\section{Upute za puštanje u pogon}
		
			Da bi se backend pripremio za deploy na Render, prvo treba provjeriti jesu li postavljene env. varijable u konfiguraciju vašeg IDE-a i po potrebi ih dodati. Potrebno je dodati i Dockerfile koji ima dvije dostupne verzije koje se nalaze u direktoriju docker. Jedna se nalazi u poddirektoriju Maven dok se druga nalazi u poddirektoriju Gradle. Ukoliko se mijenja lokacija Dockerfile-a, treba paziti na putanje \texttt{COPY} naredbi u Dockerfile skripti.\par
U datoteci application.properties preporuča se postaviti svojstvo \texttt{server.\allowbreak servlet.\allowbreak context-path} na \texttt{/api} kao prefiks svim zahtjevima na backend. Opcionalno je u application.properties dodati željene env. varijable u formatu
\texttt{neko.svojstvo=\allowbreak\$\{ENV\_VAR\_IME:\allowbreak default vrijednost\}}. Također je opcionalno u datoteci pom.xml kao dependency dodati \texttt{spring-boot-\allowbreak starter-actuator} koji će na putanju \texttt{/actuator/health} automatski izložiti informaciju o statusu aplikacije koju može koristiti Render. \hfill \break

		\noindent \textbf{Priprema frontenda za deploy na Render}

	Potrebno je u package.json dodati svaki dependency potreban za deploy, primarno \texttt{http-proxy-\allowbreak middleware}, \texttt{dotenv} i \texttt{express}. \par
	Također treba dodati \texttt{/src/setupProxy.js} koji služi kao proxy server pri lokalnom developmentu (preusmjerava API pozive na \texttt{localhost:8080}), odnosno pri  korištenju \texttt{react-scripts start} skripte. Nakon toga potrebno je dodati app.js datoteku, u kojoj se nalazi Express server za produkcijski proxy i posluživanje frontenda. Potrebno je i u datoteci package.json izmijeniti sljedeću skriptu: \texttt{"build": "yarn install \&\& react-scripts build"} i dodati \texttt{"start-prod": "node app.js"}.

		\noindent \textbf{Kreiranje baze podataka}

Nakon kreiranja Render profila, na stranici Render dashboard potrebno je pritisnuti na gumb \textbf{New +} i odabrati PostgreSQL. Zatim postaviti ime baze i opcionalno username za korisnika baze (password je automatski generiran). Pod Region odabrati Frankfurt. Odabrati željenu opciju pod Instance type. Besplatna baza podataka ima maksimalnu pohranu od 1GB, te se briše nakon 90 dana. Pritisnuti na gumb \textbf{Create Database}.
			\begin{figure}[H]
			\includegraphics[width=\textwidth]{slike/render.PNG} %veličina u odnosu na širinu linije
			\caption{Korisničko sučelje alata Render dashboard}
			\label{fig:Render} %label mora biti drugaciji za svaku sliku
			\end{figure}

		\noindent \textbf{Kreiranje backenda}

Na stranici Render dashboard potrebno je pritisnuti na gumb \textbf{New +} u gornjem desnom dijelu stranice te odabrati Web Service. U idućem koraku povezati projekt sa GitLab računom i pritisnuti \textbf{Next}, nakon čega su dostupni svi projekti na koje imate prava pristupa. Stisnuti \textbf{Connect} pored odgovarajućeg projekta. U otvorenoj konfiguraciji postaviti ime za servis koje će postati dio web-adrese.\par
Nakon toga je potrebno Root directory postaviti na \texttt{progi-be}. Zatim pod Environment odabrati Docker, a pod Region Frankfurt. Nakon toga proširiti konfiguraciju pritiskom na Advanced pri dnu stranice. \par
U toj sekciji dodati potrebne env.varijable (npr. DB username, password, URL) za čije je vrijednosti potrebno kopirati vrijdnosti iz postavki baze podataka na Renderu. Pripaziti jer URL koji je prikazan na Renderu nije JDBC URL, za ovaj primjer treba biti u formatu \texttt{jdbc:postgresql:\allowbreak//hostname:port\allowbreak/database}. Ukoliko je dodan Spring Boot Actuator (prema zadnjoj točki poglavlja pripreme za deploy), potrebno je postaviti \texttt{/api/actuator/health} kao Health Check Path (odnosno \texttt{/actuator/health}). Nakon toga postaviti putanju za Dockerfile u skladu s tim koji se package manager koristi (u ovom slučaju \texttt{./docker/maven/\allowbreak Dockerfile}). Na kraju pritisnuti gumb \textbf{Create Web Service}. \hfill \break

		\noindent \textbf{Kreiranje frontenda}

Na stranici Render dashboard potrebno je pritisnuti na gumb \textbf{New +} u gornjem desnom dijelu stranice te odabrati Web Service. U idućem koraku povezati projekt sa GitLab računom i pritisnuti \textbf{Next}, nakon čega su dostupni svi projekti na koje imate prava pristupa. Stisnuti \textbf{Connect} pored odgovarajućeg projekta. U otvorenoj konfiguraciji postaviti ime za servis koje će postati dio web-adrese. Zatim pod Environment odabrati Node, a pod Region Frankfurt. Build Command postaviti na \texttt{yarn build}, a Start Command \texttt{yarn start-prod}. Nakon toga je potrebno Root directory postaviti na \texttt{progi-be}. Nakon toga proširiti konfiguraciju psitiskom na Advanced pri dnu stranice.\par
Pod Environment variables postaviti varijablu \texttt{API\_BASE\_URL} na adresu deployanog backenda aplikacije dostupnu na Render dashboardu. Na kraju stisnuti gumb \textbf{Create Web Service}.
			
			\eject 