\chapter{Arhitektura i dizajn sustava}
		
		\textbf{\textit{dio 1. revizije}}\\

		\textit{ Potrebno je opisati stil arhitekture te identificirati: podsustave, preslikavanje na radnu platformu, spremišta podataka, mrežne protokole, globalni upravljački tok i sklopovsko-programske zahtjeve. Po točkama razraditi i popratiti odgovarajućim skicama:}
	\begin{itemize}
		\item 	\textit{izbor arhitekture temeljem principa oblikovanja pokazanih na predavanjima (objasniti zašto ste baš odabrali takvu arhitekturu)}
		\item 	\textit{organizaciju sustava s najviše razine apstrakcije (npr. klijent-poslužitelj, baza podataka, datotečni sustav, grafičko sučelje)}
		\item 	\textit{organizaciju aplikacije (npr. slojevi frontend i backend, MVC arhitektura) }		
	\end{itemize}

	
		

		

				
		\section{Baza podataka}
			
			Za potrebe našeg sustava koristit ćemo bazu podataka koja svojom strukturom olakšava modeliranje stvarnog svijeta. Osnovna jedinica baze je dokument, definiran svojim imenom i skupom atributa. Korištenjem MongoDB-a kao ne-relacijske baze podataka, informacije će biti pohranjene u obliku fleksibilnih dokumenata umjesto tradicionalnih tablica. Zadaća baze podataka je brza i jednostavna pohrana, izmjena i dohvat podataka za daljnju obradu. Baza podataka ove aplikacije sastoji se od sljedećih dokumenata:
			\begin{packed_item}
	
						\item Korisnik
						\item Događaj
						\item Korisnik-Događaj
						\item Recenzija
						\item Foto
						\item Video
						\item Obavijest	
						
						
			\end{packed_item}
		
			\subsection{Opis tablica}
			
			
				\textit{Svaku tablicu je potrebno opisati po zadanom predlošku. Lijevo se nalazi točno ime varijable u bazi podataka, u sredini se nalazi tip podataka, a desno se nalazi opis varijable. Svjetlozelenom bojom označite primarni ključ. Svjetlo plavom označite strani ključ}
				
				
				\begin{longtblr}[
					label=none,
					entry=none
					]{
						width = \textwidth,
						colspec={|X[6,l]|X[6, l]|X[20, l]|}, 
						rowhead = 1,
					} %definicija širine tablice, širine stupaca, poravnanje i broja redaka naslova tablice
					\hline \SetCell[c=3]{c}{\textbf{Korsnik}}	 \\ \hline[3pt]
					\SetCell{LightGreen}Korisničko ime & STRING	& jedinstveni identifikator korisnika\\ \hline
					Lozinka	& STRING & hash lozinke\\ \hline 
					Ime	& STRING &   ime korisnika	\\ \hline 
					Prezime & STRING &  prezime korisnika \\ \hline 
					Email & STRING	& email korisnika \\ \hline 
					Adresa & STRING & lućna adresa korisnika \\ \hline 
					Uloga & ROLE & uloga korisnika	\\ \hline 
					Gradovi interesa & LIST<GRADOVI> & gradovi interesa korisnika \\ \hline 
					Vrste interesa & LIST<VRSTE> & vrste interesa korisnika \\ \hline 
					Događanja koja me zanimaju & LIST<DOGAĐANJA> & događanja koja zanimaju korisnika \\ \hline 
					Web stranica & STRING & link web stranice organizatora \\ \hline 
					Facebook & STRING & link facebook profila organizatora \\ \hline 
				\end{longtblr}
		
			

				\textit{Svaku tablicu je potrebno opisati po zadanom predlošku. Lijevo se nalazi točno ime varijable u bazi podataka, u sredini se nalazi tip podataka, a desno se nalazi opis varijable. Svjetlozelenom bojom označite primarni ključ. Svjetlo plavom označite strani ključ}
				
				
				\begin{longtblr}[
					label=none,
					entry=none
					]{
						width = \textwidth,
						colspec={|X[6,l]|X[6, l]|X[20, l]|}, 
						rowhead = 1,
					} %definicija širine tablice, širine stupaca, poravnanje i broja redaka naslova tablice
					\hline \SetCell[c=3]{c}{\textbf{Događanje}}	 \\ \hline[3pt]
					\SetCell{LightGreen}ID & LONG &  	jedinstveni identifikator događanja\\ \hline
					Ime	& STRING &   ime događanja	\\ \hline 
					Vrsta & VRSTA &  vrsta događanja \\ \hline 
					Grad & GRAD	&  grad događanja	\\ \hline 
					Adresa & STRING	& adresa događanja \\ \hline 
					Datum & DATE & datum održavanja	\\ \hline 
					Vrijeme početka & STRING & vrijeme početka održavanja \\ \hline 
					Trajanje & STRING & trajanje događanja \\ \hline 
					Opis & STRING & opis događanja \\ \hline 
					Cijena & DOUBLE	&  cijena ulaznice	\\ \hline 
					Foto galerija & LIST<FOTO> & fotografije događanja \\ \hline 
					Video galerija & LIST<VIDEO> & video zapisi događanja \\ \hline
					Recenzije & LIST<RECENZIJE> & recenzije događanja \\ \hline
					
				\end{longtblr}
				
				
				\textit{Svaku tablicu je potrebno opisati po zadanom predlošku. Lijevo se nalazi točno ime varijable u bazi podataka, u sredini se nalazi tip podataka, a desno se nalazi opis varijable. Svjetlozelenom bojom označite primarni ključ. Svjetlo plavom označite strani ključ}
				
				
				\begin{longtblr}[
					label=none,
					entry=none
					]{
						width = \textwidth,
						colspec={|X[6,l]|X[6, l]|X[20, l]|}, 
						rowhead = 1,
					} %definicija širine tablice, širine stupaca, poravnanje i broja redaka naslova tablice
					\hline \SetCell[c=3]{c}{\textbf{Recenzija}}	 \\ \hline[3pt]
					\SetCell{LightGreen}ID & LONG	&  	jedinstveni identifikator recenzije	\\ \hline
					Sadržaj & STRING & sadržaj recenzije \\ \hline 
					Datum & DATE & datum ostavljanje recenzije  \\ \hline 
					eventLocation & VARCHAR	&  		\\ \hline 
					eventDate & VARCHAR &   	\\ \hline 
					eventStartTime & VARCHAR &   	\\ \hline 
					eventDuratin & VARCHAR &   	\\ \hline 
					description & VARCHAR &   	\\ \hline 
					gallery & VARCHAR &   	\\ \hline 
				\end{longtblr}
				
				
				
				
			
			\subsection{Dijagram baze podataka}
				\textit{ U ovom potpoglavlju potrebno je umetnuti dijagram baze podataka. Primarni i strani ključevi moraju biti označeni, a tablice povezane. Bazu podataka je potrebno normalizirati. Podsjetite se kolegija "Baze podataka".}
			
			\eject
			
			
		\section{Dijagram razreda}
		
			\textit{Potrebno je priložiti dijagram razreda s pripadajućim opisom. Zbog preglednosti je moguće dijagram razlomiti na više njih, ali moraju biti grupirani prema sličnim razinama apstrakcije i srodnim funkcionalnostima.}\\
			
			\textbf{\textit{dio 1. revizije}}\\
			
			\textit{Prilikom prve predaje projekta, potrebno je priložiti potpuno razrađen dijagram razreda vezan uz \textbf{generičku funkcionalnost} sustava. Ostale funkcionalnosti trebaju biti idejno razrađene u dijagramu sa sljedećim komponentama: nazivi razreda, nazivi metoda i vrste pristupa metodama (npr. javni, zaštićeni), nazivi atributa razreda, veze i odnosi između razreda.}\\
			
			\textbf{\textit{dio 2. revizije}}\\			
			
			\textit{Prilikom druge predaje projekta dijagram razreda i opisi moraju odgovarati stvarnom stanju implementacije}
			
			
			
			\eject
		
		\section{Dijagram stanja}
			
			
			\textbf{\textit{dio 2. revizije}}\\
			
			\textit{Potrebno je priložiti dijagram stanja i opisati ga. Dovoljan je jedan dijagram stanja koji prikazuje \textbf{značajan dio funkcionalnosti} sustava. Na primjer, stanja korisničkog sučelja i tijek korištenja neke ključne funkcionalnosti jesu značajan dio sustava, a registracija i prijava nisu. }
			
			
			\eject 
		
		\section{Dijagram aktivnosti}
			
			\textbf{\textit{dio 2. revizije}}\\
			
			 \textit{Potrebno je priložiti dijagram aktivnosti s pripadajućim opisom. Dijagram aktivnosti treba prikazivati značajan dio sustava.}
			
			\eject
		\section{Dijagram komponenti}
		
			\textbf{\textit{dio 2. revizije}}\\
		
			 \textit{Potrebno je priložiti dijagram komponenti s pripadajućim opisom. Dijagram komponenti treba prikazivati strukturu cijele aplikacije.}