\chapter{Zaključak i budući rad}
		
		Nakon 14 tjedana izrade aplikacije za objavu i traženje događanja u blizini ostvaren je zadani cilj. Većina funkcionalnosti je implementirana te radi ispravno. Tim se pokazao izrazito plodonosan i usklađen. Prema zadanim kontrolnim točkama faze rada se mogu podjeliti u dvije cjeline. 
		
		Prva faza poslužila je za dogovor oko podjele poslova i zaključak kako će se u njoj veći fokus staviti na dokumentaciju. Iz tog je rezloga tim podjeljen u tri manja podtima: backend, frontend i dokumentaciju. Definirani podtimovi nisu garantirali da će se njihovi članovi baviti samo područjem tog podtima već je donesena odluka kako će si članovi različitih podtimova međusobno pomagati kako bi svi iskusili svaki dio projekta. Tako je u ovoj fazi podtim dokumentacije dobivao konstantne povratne informacije o dogovorima oko arhitektura koje će se koristiti i na koji način. Kako bi se najkvalitetnije dogovorili svi su članovi bili prisutni na gotovo svim sastancima. 
		
		U drugoj je fazi stavljen naglasak na programskom rješenju te su se svi članovi aktivnije počeli baviti razvijanjem sustava u praksi. Kako bi svi znali što treba raditi poslužio im je plan iz dokumentacije sastavljen u prvoj fazi. U ovoj je fazi došlo do promjene načina komunikacije, projekt više nije zahtjevao prisutnost svih članova od jednom već bi se članovi međusobno savjetovali kako nebi prelazili preko tuđih rješenja. Dokumentacija se u ovoj fazi radila paralelno s razvojem programskog rješenja jer su određeni dijelovi zahtjevali testiranje samog rješenja. 
		
		Komunikacija se pokazala kao ključan faktor u uspješnosti i efikasnosti izrade projekta. Za uspješno uspostavljenu komunikaciju treba dati posebne pohvale voditelju grupe koji je kvalitetno podjelio zadatke i uvijek bio na usluzi kada je situacija to zahtjevala. Svi su članovi bili zainteresirani i odlučni u odluci da se projekt napravi na što bolji način. 
		
		Problem na koji su gotovo svi članovi naišli bio je manjak iskustva u korištenju odabranih alata. Manjak iskustva bio je vidljiv i u samom upravljanju projektom gdje je bilo potrebno vrijeme da svaki član ustanovi koja je točno njegova uloga u timu.
		
		Radi manjka vremena neke zahtjevnije funkcionalnosti nisu mogle biti implementirane no zadatak je gotovo u potpunosti obavljen. Među funkcionalnostima koje nisu implementiorane nalaze se pretplaćivanje na organizatora i objava videozapisa.
		
		Tim nije pokazivao naznake ne slaganja i loših odnosa međutim odluka o nastavku razvijanja aplikacije nije donesena. Tim je zaključio kako je sudjelovanje na projektu bilo vrijedno iskustvo za daljnji rad no kako sama ideja aplikacije nema primjenjivu namjenu koju bi članovi htijeli poduprijeti. Tim je kolektivno zadovoljan s postignutim razultatom i ne odbacuje mogućnost buduće suradnje.
		