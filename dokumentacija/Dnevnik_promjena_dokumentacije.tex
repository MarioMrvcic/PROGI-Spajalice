\chapter{Dnevnik promjena dokumentacije}
		
				
		
		\begin{longtblr}[
				label=none
			]{
				width = \textwidth, 
				colspec={|X[2]|X[13]|X[3]|X[3]|}, 
				rowhead = 1
			}
			\hline
			\textbf{Rev.}	& \textbf{Opis promjene/dodatka} & \textbf{Autori} & \textbf{Datum}\\[3pt] \hline
			0.1 & Napravljen predložak	& Josip Duvančić & 30.10.2023. 		\\[3pt] \hline 
			0.2	& Dodani funkcionalni zahtjevi & Josip Duvančić & 02.11.2023. 	\\[3pt] \hline 
			0.3 & Dodan opis projektnog zadatka & Istok Korkut & 07.11.2013. \\[3pt] \hline 
			0.4 & Dodana većina obrazaca uporabe & Josip Duvančić & 07.11.2023. \\[3pt] \hline 
			0.4.1 & Dodani svi obrasci uporabe & Josip Duvančić & 08.11.2023. \\[3pt] \hline 
			0.5 & Napravljeni dijagrami obrazaca uporabe  & Josip Duvančić & 09.11.2023. \\[3pt] \hline 
			0.6 & Napisani opisi sekvencijskih dijagrama & Istok Korkut & 09.11.2023. \\[3pt] \hline 
			0.6.1 & Prepravljanje obrazaca uporabe i izrada sekvencijskih dijagrama & Josip Duvančić & 10.11.2023. \\[3pt] \hline 
			0.7 & Početak izrade baze podataka & Josip Duvančić, Mario Mrvčić & 13.11.2023. \\[3pt] \hline 
			0.7.1 & Naprevljene tablice za bazu podataka & Josip Duvančić & 15.1.2023. \\[3pt] \hline 
			0.7.2 & Dovršeni opisi tablica za bazu podataka & Josip Duvančić & 15.1.2023. \\[3pt] \hline 
			0.7.3 & Napravljen dijagram baze podataka & Josip Duvančić & 15.1.2023. \\[3pt] \hline
			0.8 & Napisan opis arhitekture sustava & Domagoj Capar, Duje Jurić & 15.1.2023. \\[3pt] \hline
			0.9 & Napravljen dijagram razreda & Istok Korkut & 16.1.2023. \\[3pt] \hline
			0.10 & Zadnje preinake u dokumentaciji & Josip Duvančić & 17.1.2023. \\[3pt] \hline
			\textbf{1.0} & Verzija samo s bitnim dijelovima za 1. ciklus & * & 17.09.2023. \\[3pt] \hline 
			
			&  &  & \\[3pt] \hline	
		\end{longtblr}
	
	
		%\textit{Moraju postojati glavne revizije dokumenata 1.0 i 2.0 na kraju prvog i drugog ciklusa. Između tih revizija mogu postojati manje revizije već prema tome kako se dokument bude nadopunjavao. Očekuje se da nakon svake značajnije promjene (dodatka, izmjene, uklanjanja dijelova teksta i popratnih grafičkih sadržaja) dokumenta se to zabilježi kao revizija. Npr., revizije unutar prvog ciklusa će imati oznake 0.1, 0.2, …, 0.9, 0.10, 0.11.. sve do konačne revizije prvog ciklusa 1.0. U drugom ciklusu se nastavlja s revizijama 1.1, 1.2, itd.}